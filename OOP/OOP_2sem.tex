\documentclass[12pt, a5paper]{article}
\usepackage[T2A]{fontenc}
\usepackage[utf8]{inputenc}
\usepackage[english,russian]{babel}
\usepackage{amsmath, amssymb, amsfonts, amsthm}
\usepackage[left=1.5cm,right=1cm,top=1.5cm,bottom=1.5cm]{geometry}
\usepackage{hyperref}
\setlength{\parindent}{0pt}
\usepackage{listings}
\usepackage{color}

\title{Конспекты по C++ \\\large Семестр 2}
\author{Марк Тюков, Кудрявцев Иван}
\date{Весна 2020 года}

\begin{document}

\definecolor{mygreen}{rgb}{0,0.6,0}
\definecolor{mygray}{rgb}{0.3,0.3,0.3}
\definecolor{mymauve}{rgb}{0.58,0,0.82}

\lstset{ 
  backgroundcolor=\color{white},   % choose the background color; you must add \usepackage{color} or \usepackage{xcolor}; should come as last argument
  basicstyle=\footnotesize,        % the size of the fonts that are used for the code
  breakatwhitespace=false,         % sets if automatic breaks should only happen at whitespace
  breaklines=true,                 % sets automatic line breaking
  captionpos=b,                    % sets the caption-position to bottom
  commentstyle=\color{mygreen},    % comment style
  deletekeywords={...},            % if you want to delete keywords from the given language
  escapeinside={\%*}{*)},          % if you want to add LaTeX within your code
  extendedchars=\true,              % lets you use non-ASCII characters; for 8-bits encodings only, does not work with UTF-8
  frame=single,	                   % adds a frame around the code
  keepspaces=true,                 % keeps spaces in text, useful for keeping indentation of code (possibly needs columns=flexible)
  keywordstyle=\color{blue},       % keyword style
  language=C++,                 % the language of the code
  morekeywords={*,...},            % if you want to add more keywords to the set
  numbers=left,                    % where to put the line-numbers; possible values are (none, left, right)
  numbersep=8pt,                   % how far the line-numbers are from the code
  numberstyle=\color{mygray}, % the style that is used for the line-numbers
  rulecolor=\color{mygray},         % if not set, the frame-color may be changed on line-breaks within not-black text (e.g. comments (green here))
  showspaces=false,                % show spaces everywhere adding particular underscores; it overrides 'showstringspaces'
  showstringspaces=false,          % underline spaces within strings only
  showtabs=false,                  % show tabs within strings adding particular underscores
  stepnumber=1,                    % the step between two line-numbers. If it's 1, each line will be numbered
  stringstyle=\color{mymauve},     % string literal style
  tabsize=4,	                   % sets default tabsize to 2 spaces
  title=\lstname                   % show the filename of files included with \lstinputlisting; also try caption instead of title
}

%%
\maketitle
\tableofcontents
\newpage
%%

\section{Введение в язык}
\subsection{Инструкции}

\subsubsection{Declarations (объявления)}

\textbf{Переменные}
\begin{lstlisting}
type id [= value];
	
\end{lstlisting}
\
\textbf{Примеры}
ключается в стэк
[unsigned] int/long long/char/float/double/long double/bool

P.S.: size\_t $\equiv$ unsigned long long

\

\textbf{Функции, структуры}

\begin{lstlisting}	
void f(int x, double y){}
struct S{};
\end{lstlisting}

P.S.: using vi = std::vector<int>;

\paragraph{Declaration vs definition}

!!! One definition rule (ODR)

\subsubsection{Expressions}

\textbf{Базовые операторы} 

\begin{enumerate}
	\item Арифметические операторы (+, -, $\ast$, /, $\%$) 
	\item Побитовые операторы ($\&$, |, $\wedge$, $\sim$, $\ll$, $\gg$)
	\item Логические операторы ($\&\&$, ||, !)
	\item Операторы сравнения (==, <, >, <=, >=)
%% ВСТАВИТЬ ПРО Lvalue И Rvalue
	\item Assignments ($=$, $+=$, $-=$, $\ast=$, $/=$, $\%=$, $\&=$, $|=$, $\space\wedge$=, $\ll=$, $\gg=$)
	\item Инкремент и декремент ($++x$, $x++$)
	\item sizeof() - возвращает число, которое нужно для хранения входных данных (в байтах). Он не сохраняет результат операций. Например, $sizeof(x++)$ не изменит $x$
	\item Тернарный оператор "... ? ... : ..."
	\item Запятая -- выполняет левую часть, потом правую, возвращает правую. Она гарантирует, что левая часть закончит выполнение до того, как начнет выполняться правая. 
\end{enumerate} 

\subsubsection{Control statements}

\begin{enumerate}
	\item if (statement) {} else {}
	\item while(statement){}
	\item for (declaration or expression; bool expression; expression){}
\end{enumerate}

\subsection{Ошибки компиляции CE}
\subsubsection{Лексические ошибки} неизвестный символ
\subsubsection{Синтаксические ошибки} \texttt{if = 5)}
\subsubsection{Семантические ошибки} ...

\subsection{Ошибки выполнения RE - Runtime Error}
\subsubsection{Segmentation fault - обращаемся к чужой памяти} 
\begin{enumerate}
\item заканчивается стек рекурсии
\end{enumerate}

\subsection{Undefined Behaviour} 
Когда пишем что-то такое, на что компилятор в стандарте не имеет четкой инструкции.

\begin{enumerate}
\item обращение к массиву по несуществующему индексу. 
\begin{lstlisting}
int a[10];
a[100]; 
\end{lstlisting}
	В таком случае можно получить как RE, так и UB 
\end{enumerate}

\textbf{Если в программе случился UB, то не гарантируется ничего}! 
\\

	(1 != 0) будет true


\subsection{Linking error - ошибки линковщика}

После компиляции, например что-то не объявлено.
\\
\\
\textbf{Насколько плохи все эти ошибки?} 
\\
\textbf{CE} - компилятор наш друг
\\
\textbf{RE - плохо}, сервер может упасть/ может случиться что-то плохое во ВРЕМЯ ВЫПОЛНЕНИЯ
\\
\textbf{UB - ужасно!} не совершайте преступление, не делайте UB!
\\
\textit{UB и RE компилятор не блокирует зачастую при компиляции.
}
\section{Указатели. Массивы}
\subsection{Pointers, Arrays, functions, etc}
\subsubsection{Pointers}

\begin{lstlisting}
int x
{
	int y; // выделение памяти
} // - удаление из памяти (на самом деле потеря адреса. Что происходит с данными по тому адресу - хз)

type* p; 
*p; // - унарная звездочка разыменовывания.

type* p = &x; // - кладем в p адрес x
p+1; p-1;
\end{lstlisting}

\subsubsection{Операции с указателями}
\begin{enumerate}
\item Разыменовывание
\item Сложение с числами
\item Разность указателей
\end{enumerate}

\textbf{void*} - указатель на несуществующую область памяти. Такие указатели нельзя увеличивать и вычитать друг из друга

\begin{center}
Но! указатели \textbf{можно преобразовывать}
\end{center}

\textbf{nullptr} - константный указатель (типа \textit{nullptr\_t}) - аналог нуля для указателей

\texttt{*nullptr // - UB :)}

\subsection{Arrays}

type a[10] - выделение памяти на стэке для 10 элементов

\subsubsection{Операции над массивами:} 
\begin{enumerate}
\item *(a+i) Возьми адрес, где начинается массив, прибавь к нему число i и разыменуй.
\item Array to pointer conversion 
\\
\texttt{type* b = a;}
\item \texttt{sizeof(a) = } размер массива * \texttt{sizeof(тип)}
\\
\texttt{sizeof(type*) != sizeof(a);}
\end{enumerate}

\subsection{Functions}

\textbf{Сигнатура} - набор типов принимаемых аргументов.
\\
Можно объявить несколько функций с разными сигнатурами
\begin{lstlisting}
type f(int);
type f(double);
type f(int, int);
\end{lstlisting}
\begin{center}
Эти функции могут возвращать данные разных типов
\end{center}

Компилятор при вызове таких функция совершает разрешение перегрузки \textbf{(overloading resolution)} - принятие решение о выборе версии функции

\subsubsection{Overloading resolution}
\begin{enumerate}
\item В точности такой тип
\item Преобразование в int
\item Если не получается однозначно выбрать, будет ошибка компиляции \textbf{Ambiguous call}
\end{enumerate}

Пример f(float) вызываем, когда есть только от double и от int
\begin{center}
\textbf{Читать в стандарте!!!}
\end{center}

\subsubsection{Указатель на функцию}

\begin{lstlisting}
int f(double);
int (*)(double) pf = f;

type f(){
}
\end{lstlisting}

\textbf{P.S.} Запятая при указании аргументов - устойчивая конструкция для аргументов, а не \textit{Expressions}
\\
\texttt{int a = 5;} 
\\
Здесь знак равно - это \textbf{не оператор присваивания}, а устойчивая конструкция инициализации!

\subsubsection{Аргументы по умолчанию}
Аргументы по умолчанию функций указываются \textbf{в конце}
\\ \texttt{f(double x, int n = 10)}

\subsubsection{Inline}
Непосредственная \textbf{вставка кода} в указанное место \textbf{при компиляции}
\\
\texttt{inline} - лишь \textbf{рекомендации} компилятору (компилятор решает сам, он умный)


\section{Статическая и динамическая память}
\subsection{Cтатическая память}
\subsubsection{static variables}
\textbf{Свойства}
\begin{enumerate}
\item один раз заводятся
\item размер вычисляет компилятор до запуска программы
\item инициализируются один раз
\item значение при разных вызовах функции сохраняются
\end{enumerate}


\subsection{Динамическая память - выдается по требованию}
\subsubsection{операторы}
\texttt{new, delete}
\subsubsection{Переменная}
\texttt{type* p = new type();} - запрашиваем память
\\
потом нужно \textbf{освободить}
\\
\texttt{delete p;}

\subsubsection{Массив}
\texttt{type* p = new type[n];} - запрашиваем память
\\
\texttt{delete[] p;}

\begin{center}
delete и new - expressions
\end{center}

\subsubsection{Некорректоное использование delete}
\begin{enumerate}
\item Delete не на тот указатель - UB
\item delete без [] для массивов - UB
\item Если не освобождать память возможны \textbf{memory leak}
\item Если дважды удалить, то будет RE(Seg.F.)
\end{enumerate}

\textbf{P.S.}
\\
\texttt{delete p, pp;}
\\
Это \textbf{expression}. Парсится по запятой. Выполняется \texttt{delete p}. Потом \texttt{pp}.pp Не удалится ;

\section{Ссылки}
\subsection{Создание ссылки}

\texttt{type x;}
\\
\texttt{type\& y = x;} - Новое название переменной. y - \textbf{ссылка} на x.

\begin{lstlisting}
swap(int x /* - создаем локальную КОПИЮ икса */, int y){
	int t = x;
	x = y;
	y = t;
} // - так не работает.

type x;
type y = x; // - Создаем ссылку. НО НЕ В С++. В Java - да 

type x;
type& y = x; - Новое название переменной. Y - ссылка на x. 
\end{lstlisting}
\begin{center}
Отныне \textbf{нет способа отличить} y от x. Отныне и \textbf{навсегда}

\textit{"Я поступил на физтех" = "Я поступил в МФТИ"}

fewnewkrlie


\end{center}
\end{document}













